\documentclass[a4paper,twoside]{article}

% baseline packages
\usepackage[top=2.5cm, bottom=2.5cm, left=2.5cm, right=2.5cm]{geometry}
\usepackage{graphicx}
\DeclareGraphicsExtensions{.pdf,.png,.jpg,.mps,.eps,.ps}
\graphicspath{{figures/}{.}}
\usepackage[numbers,sort&compress,super,comma]{natbib}
\usepackage[table]{xcolor}
\usepackage{booktabs} % professional-quality tables
\usepackage{multirow}
\usepackage{enumitem}
\usepackage{hyperref} % for clickable links
\hypersetup{
    colorlinks = true,
    citecolor = .,
    linkcolor = .
    }
\usepackage{url}

% make it pretty
\usepackage{microtype} % improve typography
\usepackage[T1]{fontenc} % T1 encoding helps proper hyphenation and text rendering in European languages (without this OT1 is default)
\usepackage[default]{lato} % make Lato the default font
\usepackage{fancyhdr}
\usepackage{titling}

\pagestyle{fancy}

% Horizontal line settings
\renewcommand{\headrulewidth}{2pt}
\renewcommand{\footrulewidth}{1pt}

% Header and footer setup
\fancyhead{} % clear all fields
\fancyhead[C]{\emph{internal notes}}
\fancyhead[L]{Park et. al.}
\fancyfoot[C]{\thepage}

% Title setup with affiliations
\pretitle{\begin{center}\LARGE}
\posttitle{\end{center}}
\preauthor{\begin{center}
\large \lineskip 0.5em}
\postauthor{\par\end{center}
\begin{center}
\textit{
$^1$Champalimaud Centre for the Unknown, Champalimaud Foundation, Lisbon, Portugal
}
%\\
%$^\ast$corresponding author, \url{memming.park@research.fchampalimaud.org}
\end{center}}

\title{Deriving measures of dissimilarity between systems}
\author{
I.~Memming~Park$^{1}$ and
Abel S\'agodi$^{1}$
}
\date{\today}

% math & font packages
\usepackage[fleqn]{amsmath} % align equations left
\usepackage{amssymb}
\usepackage{mathtools} % \DeclarePairedDelimiter
\usepackage{bm} % bold greek

% cosmetics
\usepackage{letltxmacro}
\LetLtxMacro{\originaleqref}{\eqref}
\renewcommand{\eqref}{Eq.~\originaleqref}

%%%%%%%%%%%%%%%%%%%%%%%%%%%%%%%%%%%%%%%%%%%%%%%%%%%%%%%%
% bold vectors for each alphabet vx
\usepackage{forloop}
\newcommand{\defvec}[1]{\expandafter\newcommand\csname v#1\endcsname{{\mathbf{#1}}}}
\newcounter{ct}
\forLoop{1}{26}{ct}{
    \edef\letter{\alph{ct}}
    \expandafter\defvec\letter
}

% captial \vA
\forLoop{1}{26}{ct}{
    \edef\letter{\Alph{ct}}
    \expandafter\defvec\letter
}

%%%%%%%%%%%%%%%%%%%%%%%%%%%%%%%%%%%%%%%%%%%%%%%%%%%%%%%%
% Automatically make all greek letters bold by prepending 'v'
\usepackage{expl3}

\ExplSyntaxOn
% Define a sequence of Greek letters
\cs_new:Nn \define_bold_greek:
 {
  \seq_map_inline:Nn \g_greek_letters_seq
   {
    \cs_new:cpn { v##1 } { \bm { \csname ##1 \endcsname } }
   }
 }

% Initialize the sequence of Greek letters
\seq_new:N \g_greek_letters_seq
\seq_set_split:Nnn \g_greek_letters_seq { , }
 {
  alpha,beta,gamma,delta,epsilon,zeta,eta,theta,iota,kappa,lambda,
  mu,nu,xi,pi,rho,sigma,tau,upsilon,phi,chi,psi,omega,
  Delta
 }

% Call the function to define bold Greek macros
\define_bold_greek:
\ExplSyntaxOff
%%%%%%%%%%%%%%%%%%%%%%%%%%%%%%%%%%%%%%%%%%%%%%%%%%%%%%%%

% Math commands
\DeclareMathOperator{\diag}{diag}
\DeclareMathOperator*{\tr}{tr} % trace
\DeclareMathOperator*{\var}{var}
\DeclareMathOperator*{\cov}{cov}

\DeclarePairedDelimiter{\abs}{\lvert}{\rvert}
\DeclarePairedDelimiter{\norm}{\lVert}{\rVert}

\newcommand{\inv}{^{-1}}
\newcommand{\dm}[1]{\ensuremath{\mathrm{d}{#1}}} % dx dy dz dmu
\newcommand{\RN}[2]{\frac{\dm{#1}}{\dm{#2}}} % (Radon-Nikodym) derivative
\newcommand{\PD}[2]{\frac{\partial #1}{\partial #2}} % partial derivative

\newcommand{\field}[1]{\ensuremath{\mathbb{#1}}}
\newcommand{\reals}{\field{R}}
\newcommand{\complex}{\field{C}}
\newcommand{\integers}{\field{Z}}
\newcommand{\naturalNumbers}{\field{N}}
\newcommand{\trp}{{^\top}} % transpose
\newcommand{\identity}{\ensuremath{\mathbb{I}}}
\newcommand{\ones}{\ensuremath{\mathbf{1}}}
\newcommand{\indicator}[1]{\mathbf{1}_{#1}} % indicator function
\newcommand{\stateSpace}{\reals^n}
\newcommand{\DF}{\nabla_{\vx}\vf} %\newcommand{\DF}{\mathcal{D}\vf}

%%%%%%%%%%%%%%%%%%%%%%%%%%%%%%%%%%%%%%%%%%%%%%%%%%%%%%%%
\definecolor{mpcolor}{rgb}{1, 0.1, 0.59}
\newcommand{\TODO}[1]{(\textbf{TODO:\ }\textcolor{mpcolor}{#1})}

\definecolor{c:adjointidx}{rgb}{0.15,0.55,0.7}
\definecolor{c:time}{rgb}{0.50,0.12,0.7}
\newcommand{\homeo}{\vh}
\newcommand{\invhomeo}{\homeo\inv}

%%%%%%%%%%%%%%%%%%%%%%%%%%%%%%%%%%%%%%%%%%%%%%%%%%%%%%%%
\begin{document}
\maketitle
\thispagestyle{fancy}

\section{Background}
Let us consider two autonomous dynamical systems defined on the same space:
\begin{align}
    \dot{\vx} &= \vf(\vx)
    \\
    \dot{\vx} &= \vg(\vx)
\end{align}

If the two systems are topologically conjugate, there exists a homeomorphism $\homeo$ that maps orbits of one system to the other. Furthermore, if the flow of time is preserved, they are topologically equivalent.

Let $\homeo$ be a diffeomorphism (which means $\homeo$ is a smooth homeomorphism).
\begin{align}
    \vy &= \homeo(\vx)
    \\
    \dot{\vy} &=
	\nabla\homeo(\vx)\dot{\vx}
    =
	\nabla\homeo(\vx) \vf(\vx)
    =
	\nabla\homeo(\invhomeo(\vy)) \vf(\invhomeo(\vy))
    \\
    \dot{\vx} &= \nabla\homeo(\vx)\inv \dot{\vy}
\end{align}
where $\nabla$ is the Jacobian operator, and $\nabla\homeo(\cdot)$ is a matrix function.
The two systems are \emph{smoothly equivalent} if
\begin{align}
    \dot{\vx} &= \nabla\homeo(\vx)\inv \vg(\homeo(\vx)) = \vf(\vx)
\end{align}

Recall that a vector field is called \emph{structurally stable} if all vector fields in its neighborhood (in $C^1$ topology), are topologically equivalent to the given vector field~\cite{Chicone2006}.

\section{Distortions, Perturbations, Deformation, or Imperfections}
Let us consider two kinds of distortions.
First we have additive vector field perturbation by $\vDelta(\cdot)$.
\begin{align}
    \dot{\vx} &= \vf(\vx) + \vDelta(\vx)
\end{align}
For a finite time horizon $T$ and assuming a Lipschitz bound $L$ for $\vf$ in the relevant domain, a uniform bound on $\norm{\vDelta(\cdot)}$, we can bound the trajectory deviation from the unperturbed system.
By Gr\"onwall inequality\cite{Howard2025}:
\begin{align}
    \norm{
	\phi(\vf, \vx_0, t)
	-
	\phi(\vg, \vx_0, t)
    }
    &\leq
	e^{Lt} \int_0^t e^{-Ls} \norm{\vDelta(s)} \mathrm{d}s
\\
    &\leq
	(\max_s \norm{\vDelta(s)}) e^{Lt} \int_0^t e^{-Ls} \mathrm{d}s
    =
	(\max_s \norm{\vDelta(s)}) \frac{1}{L} (e^{Lt} - 1)
\end{align}

\bibliographystyle{unsrtnat_IMP_v1}
\bibliography{../all_ref.bib,imprefs,../catniplab.bib}
\end{document}
