\documentclass{article}
%\usepackage{arxiv}
\usepackage[numbers,sort&compress,super,comma]{natbib} % this must be before neurips_2024
\usepackage[preprint]{neurips_2024}
\usepackage[utf8]{inputenc}
\usepackage[english]{babel}
\usepackage{amssymb, amsmath, amsthm, amsfonts}
\usepackage{thmtools, mathtools, mathrsfs, dsfont}
\usepackage{bbm}
\usepackage{forloop}
\usepackage[pdftex]{graphicx}  %remove demo option in your document
\usepackage{sidecap}
\PassOptionsToPackage{dvipsnames}{xcolor}
\usepackage{xcolor}
\definecolor{ForestGreen}{rgb}{0.13, 0.55, 0.13}
\definecolor{MidnightBlue}{rgb}{0.1, 0.1, 0.44}
\definecolor{BurntOrange}{rgb}{0.8, 0.33, 0.0}
\definecolor{Plum}{rgb}{0.56, 0.27, 0.52}
\usepackage[colorlinks=true,linkcolor=MidnightBlue,citecolor=ForestGreen,filecolor=TealBlue,urlcolor=Plum]{hyperref}
\hypersetup{breaklinks=true}

\graphicspath{{figures}}

\newtheorem{theorem}{Theorem}
\newtheorem{proposition}{Proposition}
\newtheorem{lemma}{Lemma}
\newtheorem{corollary}{Corollary}[theorem]
\theoremstyle{definition} \newtheorem{definition}{Definition}
\theoremstyle{remark} \newtheorem{remark}{Remark}

\newcommand{\reals}{\mathbb{R}}
\newcommand{\manifold}{\mathcal{M}}
\newcommand{\cl}{\operatorname{cl}}
\newcommand{\relu}{\operatorname{ReLU}}
\newcommand{\vol}{\operatorname{vol}}
\newcommand{\boa}{\operatorname{BoA}}
\newcommand{\T}{\operatorname{T}}
\newcommand{\Hpert}{H^{\text{pert}}}
\newcommand{\inv}{\operatorname{Inv}}
\newcommand{\spec}{\operatorname{spec}}
\newcommand{\sign}{\operatorname{sign}}

\newcommand{\defvec}[1]{\expandafter\newcommand\csname v#1\endcsname{{\mathbf{#1}}}}
\newcommand{\dm}[1]{\ensuremath{\mathrm{d}{#1}}} % dx dy dz dmu
\newcounter{ct}
\forLoop{1}{26}{ct}{
    \edef\letter{\alph{ct}}
    \expandafter\defvec\letter
}
% captial \vA
\forLoop{1}{26}{ct}{
    \edef\letter{\Alph{ct}}
    \expandafter\defvec\letter
}

\title{Plasticity for a perturbed continuous attractor}
\author{\'Abel S\'agodi}
\date{\today}



\begin{document}
\maketitle

\section{Problem setting}

Consider 
\begin{equation}\label{eq:ode}
\tau_x\dot x = f(x) 
\end{equation}
implementing a normally hyperbolic continuous attractor, i.e., there is $\manifold_0\coloneqq \{x \in \reals^N | f(x) = 0 \text{ and } \spec_{i>1}(J(x)) <0\}$  (all eigenvalues except for one of the Jacobian at are negative). 

Now imagine that this system has been affected so that the system has an approximate continuous attractor
\begin{equation}\label{eq:ode_pert}
\tau_x\dot x = f(x)  + \epsilon p(x).
\end{equation}

Then the flow on the perturbed persistent invariant manifold $\manifold_\epsilon$ is
\begin{equation}\label{eq:ode_pert}
\tau_x\dot x = \epsilon p'(x).
\end{equation}
%how are p and p' related?
We can say that they close to each other, related to $\mathcal{O}(\epsilon)$ and the tangent/derivative of the of f(x) on $\manifold_\epsilon$ being $C^1$ close $\manifold_0$.



\section{Plasticity/Homeostasis/Adaptation}
%consider bias pert? 

Let's analyze $f(x,W_0,t) = -x(t) + W_0\relu(x(t)) + b$ with $W_0\in\reals^{n\times n}$ and $b\in\reals^n$ so that the system implements a ring attractor \citep{noorman2024accurate}.

We consider the case where the connectivity has been affected: $W(t=0) \leftarrow W_0 + \epsilon V$.
In this case we simply have $g(x,W,t) = (W(t)-W_0)\relu(x(t))$.

$g_i(x,W,t) = \sum_{j}^N(W_{ij}(t)-W_{0,ij})\relu(x_j(t))$.

Now, let's assume that $x(t)\in\manifold_\epsilon$.
Define the plasticity rule
\begin{equation}\label{eq:plasticity}
\tau_W\dot W_{ij}(t) = -\epsilon \sign(g_j(x,W,t)) \phi_j(x) = -\epsilon \sign(\dot x_i) \relu(x_j) - \gamma W_{ij}.
\end{equation}
%should this have a factor \|\sign(g_j(x,W,t))\| to make it continuous?

%another attempt through finding a gradient desecnt on the weights
%\begin{equation}\label{eq:plasticity}
%\tau_W\dot W_{ij}(t) = -\epsilon\frac{\partial g(x,W)}{\partial W_{ij}}.
%\end{equation}
%For the above neural dynamics we simply have
%\[
%\frac{\partial g(x,W)}{\partial W_{ij}} = \begin{cases} x_j  & \text{ if } x_j > 0 \\ 0 & \text{ otherwise.} 
%\end{cases}
%\]



\subsection{Contractivity}
%weight decay necessary for contractivity ananlysis?

Plasticity analysis through contractivity \citep{kozachkov2022matrix} \citep{centorrino2024modeling}.









\newpage
\bibliographystyle{plain}
\bibliography{../../all_ref.bib}

\end{document}