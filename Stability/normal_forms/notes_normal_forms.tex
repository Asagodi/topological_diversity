%! LW recipe=tectonic
\documentclass{scrartcl}
\usepackage{tcb-callouts}
\usepackage[notcb]{tcb-theorems}
\usepackage[english]{babel}
\usepackage{csquotes,graphicx,enumitem,url,hyperref}
\usepackage[natbib=true,
  sortcites=true,
  url=false,
  maxcitenames=2,
  backref=true]{biblatex}
\hypersetup{bookmarksopen = true,
  bookmarksnumbered=true,
}
\addbibresource{zotero.bib}

\title{Normal forms for ring attractors and their perturbations}
\author{}
\date{}

\begin{document}
  \maketitle
  \section{The purpose}
    Using group theory we can write down pretty general results about compact continuous attractors, and their perturbations.
    Given our interest in ring attractors, we'll focus on those, plus they have the advantage of being particularly tractable.
  \section{Normal forms for O2 and SO2 symmetry}
    \begin{example}[{\cites[Ex.~1.33]{golubitsky2002symmetry}}]\label{notes_normal_forms:ex:1.33}
      Let \(\Gamma=\mathbf{O}(2)\) act on \(\mathbb{R}^{2} \cong \mathbb{C}\).
      The action on \(\mathbb{C}\) is generated by:
      \begin{equation*}
        \theta z \equiv e^{i \theta} z \quad \theta \in[0, 2 \pi)
      \end{equation*}
      and
      \begin{equation*}
        \kappa z=\bar{z}
      \end{equation*}
      For \(\Sigma=\mathbb{Z}_{2}(\kappa)=\{1, \kappa\}\) we have \(\operatorname{Fix}(\Sigma)=\mathbb{R}\).
      Thus, at an \(\mathbf{O}(2)\) steady-state bifurcation, generically, there exist equilibria with a reflectional symmetry.
    \end{example}
    \begin{example}[{\cites[Ex.~1.34]{golubitsky2002symmetry}}]\label{notes_normal_forms:ex:1.34}
      Let \(\Gamma=\mathbf{D}_{m}\) act on \(\mathbb{C}=\mathbb{R}^{2}\).
      The action on \(\mathbb{C}\) is generated by:
      \begin{equation*}
        \theta z \equiv e^{i \theta} z, \quad \theta = 2 \pi / m
      \end{equation*}
      and
      \begin{equation*}
        \kappa z=\bar{z}
      \end{equation*}
      For \(\Sigma=\mathbb{Z}_{2}(\kappa)\) we have \(\operatorname{Fix}(\Sigma)=\mathbb{R}\), and there exist solutions with reflectional symmetry.
      Next consider \(\mathbf{D}_{m}\) symmetry for \(m = 4\) and 5 .
      (Note that \(m = 4\) is typical of even \(m\) and \(m = 5\) is typical of odd \(m\).)
      Group related solutions have conjugate isotropy subgroups.
      If \(f(v, \lambda)=0\), then \(f(\gamma v, \lambda)=\gamma f(v, \lambda)=0\).
      Suppose \(\sigma v = v\) then \(\left(\gamma \sigma \gamma^{-1}\right) \gamma v=\gamma \sigma v=\gamma v\).
      So
      \begin{equation*}
        \Sigma_{\gamma v}=\gamma \Sigma_{v} \gamma^{-1}
      \end{equation*}
      Also if \(T=\gamma \Sigma \gamma^{-1}\) then \(\operatorname{Fix}(T)=\gamma(\operatorname{Fix}(\Sigma))\).
      So when \(m = 5\) there is a single conjugacy class of isotropy subgroups.
      However when \(m = 4\) there are two conjugacy classes.
      The bifurcation diagrams are given in Figure 1.5.
      To verify this statement note that when \(m = 4\) there are two types of lines of symmetry: those connecting vertices and those connecting midpoints of opposite sides.
      This is the geometric realization of the two conjugacy classes.
      When \(m = 5\) there is only one type of line of symmetry.
    \end{example}
    \begin{example}[{\cites[Ex.~4.4]{golubitsky2002symmetry}}]\label{notes_normal_forms:ex:4.4}
      The standard actions of \(\mathbf{S O}(2)\) on \(\mathbb{R}^{2}\) and of \(\mathbb{Z}_{q}(q \geq 3)\) on \(\mathbb{R}^{2}\) are irreducible but not absolutely irreducible.
      However \(\mathbf{O}(2)\) and \(\mathbf{D}_{q}\) act absolutely irreducibly.
    \end{example}
    \begin{example}\label{golubitsky2000symmetry:ex:6.1}
      Let \(\Gamma=\mathbf{O}(2)\) with its standard action on \(\mathbb{R}^{2} \equiv \mathbb{C}\).
      Every equivariant vector field is of the form \(g\left(|z|^{2}\right) z\).
      In polar coordinates \((r, \theta)\) we have
      \begin{equation*}
        \dot{r}=r g\left(r^{2}\right) \quad \dot{\theta}=0
      \end{equation*}
      reflecting the fact that the vector field is purely radial.
      This is a consequence of equivariance under the reflections in \(\mathbf{O}(2)\).
      Since \(\theta\) is constant, we can restrict attention to the \(r\)-equation only.
      Relative equilibria are given by the zeros of \(g\).
      In this case, a relative equilibrium is an \(\mathbf{O}\) (2) group orbit of equilibria and equilibria other than the origin occur in circles.
    \end{example}
    \begin{example}\label{golubitsky2000symmetry:ex:6.2}
      The \(\mathbf{S O}(2)\) case is different, and in an interesting way.
      Now we let \(\Gamma=\mathbf{S O}(2)\) with its standard action on \(\mathbb{R}^{2} \equiv \mathbb{C}\).
      Every equivariant vector field is of the form \(g\left(|z|^{2}\right) z + h\left(|z|^{2}\right) i z\).
      In polar coordinates \((r, \theta)\) we have
      \begin{equation*}
        \dot{r}=r g\left(r^{2}\right) \quad \dot{\theta}=h\left(r^{2}\right)
      \end{equation*}
      and the vector field has tangential as well as radial components.
      Again, relative equilibria other than the origin are given by the zeros of \(g\), and relative equilibria other than the origin occur in circles ( \(\mathbf{S O}(2)\) group orbits).
      Generically, \(h\) is nonzero at a zero \(r_{0}^{2}\) of \(g\).
      It follows that the flow around the circle \(r = r_{0}\) is given by
      \begin{equation*}
        \dot{\theta}=h\left(r_{0}^{2}\right)
      \end{equation*}
      and the solution is a time-periodic rotating wave (time evolution is the same as spatial rotation).
      In this case, relative equilibria are generically time-periodic not a group orbit of equilibria.
    \end{example}
    \begin{example}\label{golubitsky2000symmetry:ex:6.13}
      Consider \(\Gamma=\mathbf{O}(2)\).
      Steady-state bifurcations lead (generically) to a branch of solutions with \(\mathbf{D}_{k}\) symmetry for some \(k \geq 1\).
      The group orbit through \(x_{0}\) of these solutions is a circle in phase space.
      In this example we consider the two simplest cases: \(\Sigma_{x_{0}}=\mathbf{D}_{1}\) and \(\Sigma_{x_{0}}=\mathbf{D}_{2}\).
      In either case the irreducible representations of \(\Sigma_{x_{0}}\) are one-dimensional: there are two possibile irreducible representations when \(\Sigma_{x_{0}}=\mathbf{D}_{1}\), and four when \(\Sigma_{x_{0}}=\mathbf{D}_{2}\).

      First, suppose that \(\Sigma_{x_{0}}=\mathbf{D}_{1}\).
      The trivial representation does not break symmetry, so we ignore it.
      The only nontrivial representation is \(x \mapsto - x\).
      Suppose that we have a group orbit of equilibria and a one-dimensional kernel of \((\mathrm{d} g)_{x_{0}}\) on which \(\Sigma_{x_{0}}\) acts nontrivially.
      Then, generically, we can suppose that the normal vector field undergoes a pitchfork bifurcation to a new steady state \(y\) with \(\Sigma_{y} = 1\).
      Since
      \begin{equation*}
        N(\mathbf{1}) / \mathbf{1} \cong \mathbf{O}(2)
      \end{equation*}
      Theorem 6.4 implies that bifurcation leads to in a rotating wave.
      In the case where \(\Sigma_{x_{0}}=\mathbf{D}_{2}\), write \(\mathbf{D}_{2}=\langle\kappa, \pi\rangle\) as usual, \(=\kappa\) being reflection and \(\pi\) being rotation.
      The kernel of the irreducible representation is either \(\mathbf{D}_{2}, \mathbb{Z}_{2}^{\kappa}, \mathbb{Z}_{2}^{\pi}\), or \(\mathbb{Z}_{2}^{\kappa \pi}\).
      The first case does not break symmetry and again we ignore it.
      If the kernel is \(\mathbb{Z}_{2}^{\kappa \pi}\) we get a rotating wave since the normalizer is \(\mathbf{O}(2)\).
      In the other two cases the normalizer is \(\mathbf{D}_{2}\), so bifurcation leads to a group orbit of equilibria.
    \end{example}
    \begin{theorem}[{\cites[Thm.~9.8]{golubitsky2002symmetry}}]\label{notes_normal_forms:thm:9.8}
      Let \(A\) be an attractor for a system of differential equations with \(\mathbf{O}(2)\) symmetry.
      Generically, either
      \begin{enumerate}
        \item[(a)]  A reflectional symmetry fixes each point in \(A\), or
        \item[(b)]  \(\mathbf{S O}(2)\) leaves A invariant.
      \end{enumerate}
    \end{theorem}
  \section{Chossat}
    \subsection{Bifurcations in R2}
    Symmetries in \(\mathbb{R}^{2}\).
    Recall that \(\mathbf{O}(2)\) (the symmetry group of the circle) is spanned by the rotation group \(\mathbf{S O}(2)\) and the reflection \(S:(x, y) \mapsto(x,-y)\).
    If a nontrivial element \(A \in \mathbf{O}(2)\) fixes a point \(M\) off the origin, \(A\) can only be the reflection through the axis \(\Delta = O M\).
    If \(\varphi\) denotes the angle between \(\Delta\) and \(O x\), then the reflection through \(\Delta\) is \(S_{\Delta}=R_{\varphi} S R_{-\varphi}\) where we denote by \(R_{\varphi}\) the rotation matrix of angle \(\varphi\).
    Given any symmetry group \(G\) acting in the plane, i.e. any closed subgroup of \(\mathbf{O}(2)\), its one-dimensional fixed-point subspaces, that is its axes of symmetry, are in one-to-one correspondence with its reflection elements.
    In fact the conjugacy class of a reflection in \(G\) with symmetry axis \(\Delta\) is in one-to-one correspondence with the \(G\)-orbit of \(\Delta\).
    Below we list the possible cases.
    \begin{itemize}
      \item \(G=\mathbf{O}(2)\).
        Any axis in \(\mathbb{R}^{2}\) is an axis of symmetry, and therefore the bifurcation problem is completely solved by applying the equivariant branching lemma to the axis \(O x\).
        Moreover the rotation by \(\pi\) clearly acts as \(\mathbf{- 1}\) in \(O x\) (and of course in any axis).
        Hence the possible bifurcations are of pitchfork type.
        More algebraically : the normalizer of the reflection \(S\) is the group spanned by \(S\) and \(R_{\pi}\), which we may write \(\mathbb{Z}_{2}(S) \times \mathbb{C}_{2}\left(R_{\pi}\right)\).
        Since \(\mathbb{Z}_{2}(S) \times \mathbb{C}_{2}\left(R_{\pi}\right) / \mathbb{Z}_{2}(S) \simeq \mathbb{C}_{2}\left(R_{\pi}\right)\), which acts by \(-\mathbf{1}\) in \(\Delta\), the bifurcation is a pitchfork.
      \item \(G=\mathbf{D}_{2 p + 1}\).
        This is the symmetry group of a regular \(n\)-gon with \(n = 2 p + 1\).
        It is generated by the rotation \(R_{2 \pi / n}\) and by the reflection \(S\).
        There is one orbit of symmetry axes, which contains \(2 p + 1\) elements.
        Therefore the equivariant branching lemma applies to only one isotropy type of solutions.
        Since \(R_{\pi}\) is not an element in \(\mathbf{D}_{2 p + 1}, N\left(\mathbb{Z}_{2}(S)\right)=\mathbb{Z}_{2}(S)\).
        Therefore the bifurcation is neither saddle-node nor pitchfork.
        It is transcritical when \(p = 1\) and onesided when \(p>1\) (this follows from the equivariant structure of the bifurcation equation, see Section 5.1.3).
      \item \(G=\mathbf{D}_{2 p}\).
        This is the symmetry group of a regular \(n\)-gon with \(n = 2 p\).
        It is generated by the rotation \(R_{2 \pi / n}\) and by the reflection \(S\).
        There are however two types of axes of symmetry in this case.
        Indeed, let us conjugate \(S\) with the rotation \(R_{\pi / 2 p}\) which does not belong to \(\mathbf{D}_{2 p}\).
        Since \(R_{\pi / 2 p} S R_{-\pi / 2 p}=R_{\pi / p} S \in \mathbf{D}_{2 p}\), this element is a reflection but is not conjugated to \(S\).
        It is easy to check that there are just two kinds of reflections (i.e. two \(\mathbf{D}_{2 p}\)-orbits).
        This corresponds for example when \(2 p = 4\) to the fact that a square has two types of symmetry axes : those joining the middle of opposite edges and those joining opposite vertices.
        In every axis however, the rotation by \(\pi\) acts as \(\mathbf{- 1}\).
        Therefore the equivariant branching lemma leads to two types of pitchfork bifurcations.
    \end{itemize}
\end{document}
