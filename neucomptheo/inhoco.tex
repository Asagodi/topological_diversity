%\documentclass{article}
\documentclass{scrartcl}
\usepackage[utf8]{inputenc}
\usepackage{amssymb, amsmath, amsthm}
\usepackage{thmtools, mathtools, mathrsfs}
\usepackage{amsfonts}
\usepackage[sort&compress,numbers]{natbib}
\usepackage{subcaption}
\usepackage{graphicx}
\usepackage{caption}
\usepackage{float}
\usepackage{bm}
\usepackage{tikz}
\usetikzlibrary{positioning,matrix,arrows,decorations.pathmorphing}
\usepackage{tikz-cd} 
\definecolor{processblue}{cmyk}{0.8,0,0,0}
\definecolor{mpcolor}{rgb}{1, 0.1, 0.59}
\newcommand{\mpcomment}[1]{(\textbf{MP:\ }\textcolor{mpcolor}{#1})}

\newcommand{\mb}[1]{\mathbb{#1}}
\newcommand{\mc}[1]{\mathcal{#1}}
\DeclareMathOperator{\Inv}{Inv}
\DeclareMathOperator{\innt}{int}
\newcommand{\probP}{\text{I\kern-0.15em P}}

\newtheorem{theorem}{Theorem}
\newtheorem{prop}{Proposition}

\theoremstyle{definition}
\newtheorem{definition}{Definition}
\theoremstyle{remark}
\newtheorem{remark}{Remark}

\newcommand{\reals}{\mathbb{R}}

 \usepackage{thmtools, thm-restate} \newtheorem{conjecture}[theorem]{Conjecture}
 \newtheorem{corollary}[theorem]{Corollary}

\title{A framework for infinite horizon neural computation on compact domains}
\author{\'Abel S\'agodi}
\date{September 1, 2023}

\begin{document}
\maketitle


\section*{Goal}
The goal is to give a framework to be able to characterize and describe all possible computations that can be implemented in neural networks with asymptotic dynamics and robustly in the language of dynamical systems (at a level that is understandable for humans).

Why: to represent the computation performed by a dynamical system throught a mapping to a finite state machine.
For asymptotic dynamics one can formulate a surjective mapping to finite-state machines (FSMs) if MSs are fixed points or one restricts what effects the input can have.
If no continua are allowed (because they are not robust) there is a surjective mapping to nondeterministic FSMs. 
But this framework is more expressive because it can deal with infinite states as long as they are on a continuum.
See Sec.\ref{sec:fsm}

\section*{Abstract}
This document presents a comprehensive framework for characterizing and describing computations realizable within the asymptotic dynamics of neural networks on compact domains in the language of dynamical systems,  leveraging Conley's decomposition theorem to elucidate the nature of flows in compact phase portraits. The scope of the possible computations corresponds to infinite horizon computations with respect to decoding time. We state theorems to describe the indispensable reliance on chain-recurrent sets in asymptotic neural computations on compact domains. Memory states are identified as the Morse sets of the system. The interplay between meaningful memory states and neural computation with linear decoders is analyzed, emphasizing the requirement of linear separability for sets of trajectories grouped by the preimage of the input. 
Additionally, we describe the syntax of neural computation, providing insights into input-driven dynamics under diverse modeling conditions. 



\newpage
\section*{The story so far}
We describe the general implementational principles of robust, timing-independent (infinite horizon) neural computation.
If a system is implementing such a computation, on a compact domain, the state will evolve to an attractor state inside the chain-recurrent recurrent set.
These states corrspond to memories: internal/intermediary (no change in output), external/behavioral (change in output when transitioning to it).\footnote{Change is characterized in terms of a complete output sequence.}

Such computation can be represented as a finite-state machine (FSM) in a lossess way (without losing anything relevant for the computation).

To describe neural computation in more generality, we construct a metric in the space of dynamical systems defined by the distance to the input-output mapping to infinite horizon computation systems.


\section*{How to proceed}

Define input space.

Define possible outputs. Constant/perdiodic vs. transient (only possible as final behavior).

Define allowable I/O mappings.


Metrics:
- ASD-FSM approximations
-- Zero for inhoco, nonzero for non-inhoco
- CA approximations 


Consequences for what can be an implementation of a (inho) computation.
- (linear) separability of basin of attraction




\section*{Open questions}

Output mapping: What can we say when $g$ is nonlinear?
- if monotonous: separability of boas is still required (?)


Inference: 


\section*{Issues}


Representation: for models in different classes (can be lossy or lossless).
Abstraction: for models in the same class but one is a lossy representation of the other.

\section*{Relevant literature}
Similar: \citep{casey1996}

Slightly related: 
\citep{}
\citep{}

\newpage
\section{Introduction to Infinite horizon computation with ODEs}	
Computation is generally equated with information-processing, and this is why the notion of information is crucial in models of computation for the account of implementation: a computational process is one that transforms the stream of information it has as input into a stream of information for output \citep{milkowski2014}.
Flexibility: Both computers and brains are general‑purpose devices: they can operate in a variety of ways, which makes them adaptable to their environment, and this, in turn, can explain intelligence. \citep{milkowski2018computermetaphor}
cognition requires not only flexibility, which in turn requires universality, but also that semantic interpretability requires universality.
 In other words, \citep{pylyshyn1984} has claimed that finite‑state machines (FSM) cannot be semantically interpretable
%This is exactly why Newell and Simon made \emph{universal computation} part of their hypothesis.
We define the implemented (neural) computation as an input-output mapping for times $t\in T\subset\reals$ defined by the function $\varphi:\reals^T\rightarrow\reals^K$:\footnote{We discuss some restrictions on the type of mapping we consider in Sec.~\ref{sec:jar}.}
\begin{equation}
o(t) = \varphi(\{u(t)\}_{t\in T})
\end{equation}
where $o(t)$ is the target output for each time point $t\in T$ and the input  $u:\reals_{+}\to \reals^{m}$  is a Lebesgue measurable essentially bounded external input.

Internal dynamics is a non-autonomous ODE, driven by input $u(t)$:
\begin{align}
\dot x(t) &= f(x(t),u(t))\label{eq:ode}\\
y(t) &= g(x(t))	\label{eq:output}
\end{align}
where $f:\reals^N\rightarrow\reals^N$, is a Lipschitz continuous function w.r.t. the first argument and uniformly w.r.t. the second one. This ensures that there exists a unique absolutely continuous solution of the system in Eq.~\ref{eq:ode}.
while the output is given a simple (linear, monotonic) mapping $g:\reals^N\rightarrow\reals^K$.


\begin{definition}
Infinite horizon: agnostic to decoding time
\end{definition}







\begin{remark}
The output can only change if there is a change in the input. 
So cued output is allowed, but timing tasks are excluded.
\end{remark}

\subsection{Justifications and restrictions}\label{sec:jar}
\subsubsection{Asymptotic}
We justify the focus on asymptotic dynamics by considering systems where the asymptotic dynamics dominates due to a significant time scale difference between stimulus statistics and the asymptotic dynamics.


We now define a discrete time representation of a continuous time dynamical system defined by an ODE
\begin{equation}\label{eq:inputdriven}
\dot x(t) = f(x(t), u(t))
\end{equation}
which in turn defines a flow $\varphi(t,x)$.

Remark: For infinite horizon computation this representation is lossless (if the output is not considered, for nonstationary output there is a loss).
Explain why?

The input sequence has the following restrictions

Because incoho should work where arbitrary silent (zero input) sequences are introduced into the original the input sequence, this restriction is natural for incoho.

The dynamics of the discrete dynamical is defined in two steps

1. The system is integrated as a input forced/input driven system (Eq.~\ref{eq:inputdriven}) until the next zero input sequence, at time $T$.

2. The $\omega$-limit set is calculated from the autonomous flow given the initial state $x_0=x(T)$.



Restrictions on the type of output: 

This corresponds to saying that non-stationary inputs are like behavioral syllables.


\subsubsection{Compact domain}
Brain activity is compact

\subsubsection{Robustness}
Brain is noisy, see also Rem.~\ref{rem:error_noise}

\paragraph{Robustness definition(s)}

\begin{definition}[Robust computation]\footnote{Based on Definition 2.2.1.in \citep{kuehn2015}}
 Let $\varphi$ and $\varphi'$ be two input-output mappings.
  Then we say that $\varphi$ is $\epsilon$-close to $\varphi'$ if
\begin{equation}\label{eq:error_def}
\sup_{u\in \mathcal{I}}\|\int_{t\in T}\varphi(t) - \int_{t\in T}\varphi'(t)\|  \leq \epsilon,
\end{equation}
\end{definition}


Minimal noise model: 
A single random delta function is applied to the asymptotic behavior
Definition:
Asymptotic dynamics is given by $x(t)$
For an arbitrary $t \in \reals$ we reinitilize the system as $x'(t)=x(t) + \theta$ where $\theta\sim \mathcal{U}(\mathcal{N}_\epsilon(M))$ (the uniform distribution over the $\epsilon$-neighbourhood of the Morse set $M=\{x(t)\colon t\in \reals\}$.

Def 1: Robustness in the light of this definition for a FSM equivalent DS can be formulated as that the error as defined in Eq.~\ref{eq:error_def} is zero.

Def 2: Robustness in the light of this definition for a FSM equivalent DS can be formulated as that the error as defined in Eq.~\ref{eq:error_def} is a continuous function of $\epsilon$.

Later: implications to what can be a robust, persistent memory state
E.g.  FSM equivalent DS: only stable fixed points as memory states (because of equivalence)
but also CAs are not allowed (because of error/perturbation effect)


\section{Theorems on infinite horizon neural computation on a compact domain}\label{sec:theorems}


\subsection{Memory states}%the words in the language
Q: What form can the states in the finite-state machine take?
A: They are the Morse sets.

Young’s account of basic memory unitsmnemons in his terminology—is not entirely loose, as the mnemon is supposed to “record one bit of information” \citep{young1978}. %p.87
Memory Units in Cephalopods (Sec.4.1 in \citep{milkowski2018})
Gifford Lectures (Young 1978), instead of framing memory in terms of static models or representations, he uses the term “program.”: expresses a concept of hierarchical and dynamical control

“programs of the brain” are related to cyclical operations of the nervous system, as driven by physiological rhythms

Young framed the computational principles of two memory systems of the octopus in terms of distributed processing in “serial networks, with recurrent circuits” (Young 1991, p. 200).

\begin{theorem}
Asymptotic neural computation that is bounded (that happens on a compact domain) has to rely on Morse sets to represent (input, internal and output states).
\end{theorem}

\begin{proof}
The asymptotic dynamics converges to the Morse sets for a compact dynamical system \cite{conley1978}, see Sec.~\ref{sec:dst}.
\end{proof}


\begin{theorem}	
Infinite horizon neural computation that is bounded has to rely on the sets that have a Morse set as their $\omega$-limit set.
\end{theorem}	

\begin{proof}
Invariance to time decoding implies that trajectories (after the input is presented) that get decoded into a certain output need to retain that decoding.
%mathematical formulation

Each trajectory converges to a Morse set asymptotically.

\begin{itemize}
\item For constant output (after last input): Each of these trajectories needs to be mapped onto the same value as the $\omega$-limit set.
\item For dynamic output: Each state along these trajectories needs to be mapped onto the same value as the $\omega$-limit set corresponding to a value along the dynamic output. %is the notion of inertial manifold useful here
\end{itemize}
\end{proof}


\subsubsection{Robust memory states}
Attractor and basin of attraction should be nonzero measure set.


\subsubsection{Continuous attractors}%and its approximations



%\subsubsection{Meaningful memory states (for the behavior) are charaterized by the output}
\subsection{Characterization of memory states}

\begin{theorem}
For neural computation with a linear decoder, the sets of trajectories grouped by input (i.e. preimages of the decoder $g^{-1}(y)$ for $y\in Y$ the output space) must be linearly separable.
\end{theorem}

\begin{remark}
If the assumption of a compact domain is relaxed, then the computation can involve an inertial manifold (as long as it is aligned with the nullspace of the decoder).
\end{remark}

We distinguish between two classes of Morse sets.
The first one can only hold a single memory state. 
The Morse sets in this class either have a periodic or quasiperiodic behavior or they contain a wandering set.
All points inside such a set need to be mapped onto the same point, i.e. for some $y\in Y$ for all $x\in M$ we have that $g(x)=y$.

The second one can hold a continuum of memory states. %These are continuous attractor-like sets.
Such sets to be used as a neural integrator we state the following constraint on the input:
There is an input value that pushes the internal state along $M$.

On the other hand if $M$ is  mapped onto a single output, then the integrated information is either not used or used only internally in the computation.

\subsection{Syntax}%how words and morphemes combine to form larger units
%how information is integrated
%how input interacts with the represented information/the internal state

\subsubsection{Input driven dynamics}
Two cases to model it:
\begin{enumerate}
\item Input is modelled as a delta function, input reinitializes the system
\item Input is continuous 
\begin{itemize}
\item If input is constant, study it as a bifurcation parameter
\item Otherwise: non-autonomous dynamics (perhaps inertial manifold can be useful, they can provide a condition when dynamics is equivalent to the dynamics of an autonomous system)
\item Consider timescale of input effect much shorter, then consider only asymptotic of input driven dynamics (but this would make neural integration of a continuous signal impossible)
\end{itemize}
\end{enumerate}


\section{Finite state machine equivalencies}\label{sec:fsm}

\subsection{FSM representation}
%FSM repr
In a synchronous circuit, an electronic oscillator called a clock (or clock generator) generates a sequence of repetitive pulses called the clock signal which is distributed to all the memory elements in the circuit. 
The ``clock'' determines the sequence of events in the following way:
\paragraph{Input driven dynamics}



\subsection{Infinite horizon computation represented as a deterministic FSM}
For the representation to be lossless, the system 
\begin{itemize}
\item has to have only fixed points as attractors in the chain-recurrent set (with the possibility of other non-stationary attractors that get mapped to the final states of the FSM).
\item  the input driven dynamics needs to be restricted in such a way that each (relevant) input has a unique effect on each attractor (Morse set).
\item the output needs to map the possible non-unique effects to a single output.
\end{itemize}


\begin{theorem}
If the sinks (final nodes in the directed acyclic graph (DAG) representation of the Morse Decomposition) of finest Morse decomposition is composed of fixed points, then there exists a surjective mapping to deterministic finite state machines.
\end{theorem}

\begin{proof}

\end{proof}

\begin{corollary}
Surjective mapping from ADSs with output to Moore machines.
\end{corollary}

\begin{corollary}
This mapping can be extended to dynamical systems where the  the final states of the FSM are non-stationary Morse sets.
\end{corollary}

All these surjective maps define equivalence relations in the space of dynamical systems that implement infinite horizon computation.


\section{Metrics}

- ASD-FSM approximations

-- Zero for inhoco, nonzero for non-inhoco




- CA approximations 
-- Discretization of CA
--- Line attractor
---- stationary asymptotic behavior
We first look at the case of a line attractor (LA) with output trajectory $x(t)=\text{constant}$ and an approximation of a line attractor (ALA) with an output trajectory $x'(t)$:
\begin{equation}
d_*(LA, ALA) \coloneqq \lim_{t\rightarrow\infty} |x(t) - x'(t)|
\end{equation}


---- any x'(t)
\begin{equation}
d_T(LA, ALA) \coloneqq \frac{1}{T} \int_0^T  |x(t) - x'(t)| dt
\end{equation}

\begin{equation}
d_\infty(LA, ALA) \coloneqq \lim_{T\rightarrow\infty} \frac{1}{T} \int_0^T  |x(t) - x'(t)| dt
\end{equation}




-- 


%%%%%%%%%%%%%%%%%%%%%%%%%%%%%%%%%%%%%
\section{Appendix}
\subsection{Dynamical systems theory}\label{sec:dst}

Conley's decomposition theorem states that every flow of a dynamical system with compact phase portrait admits a decomposition into a chain-recurrent part and a gradient-like flow part.


Given a compact invariant set $S$ is there a finest collection of subsets of $S$ off of which one can define a Lyapunov function: the chain recurrent set. Let us first introduce the notion of chain recurrence.

\begin{definition}[$(\epsilon,\tau)$ chain]
An $(\epsilon,\tau)$ chain from $x$ to $y$ is a finite sequence 
\[f(x_i, t_i) \subset  X \times  [\tau,\infty); i = 1, \dots, n\]
such that $x = x_1, t_i\geq \tau, \mu(\varphi(t_i, x_i), x_{i+1})\leq \varepsilon$   and $\mu(\varphi(t_n, x_n), y)\leq \epsilon$.
 If there exists an $(\epsilon,\tau)$ chain from $x$ to $y$, then we write $x \succeq_{(\epsilon,\tau)}y$. If  $x \succeq_{(\epsilon,\tau)}y$ for all $(\epsilon,\tau)$, then  $x \succeq y$.
\end{definition}


\begin{definition}[Chain-recurrent set]
The chain recurrent set of $X$ under the flow $\varphi$ is defined by 
\[\mathcal{R}(X,\varphi) = \{x\in X|x\succeq x\}.\]
\end{definition}

The flow is called \emph{strongly gradient-like} if the chain recurrent set is totally disconnected (and consequently equal to the rest point set). 

The recurrent set captures all the recurrent dynamics as stated and proven in \citep[Chapter I.8.2]{conley1978}.

The only type of dynamics that is outside of the recurrent set is strongly gradient-like:

Statement. Every flow on a compact space is uniquely represented as the extension of a chain recu"ent flow by a strongly gradient-like flow; that is the flow admits a unique subflow which is chain recurrent and such that the quotient flow is strongly gradient-like.

\begin{remark}\label{rem:error_noise}
%error->noise
The significance of the statement is explained by contrasting the two types (strongly gradient-like and chain recurrent) of flow, and the contrast is best seen if (arbitrarily) small errors are allowed in following solutions. For strongly gradient-like flows the following is true: if $U$ is any neighborhood of the rest point set, there exists a positive $\epsilon$ such that if the (persistent) error made in following solutions for time 1 is no larger than $\epsilon$ then all solutions eventually enter and subsequently stay in $U$.

Chain recurrent flows are just the opposite: if $\dots,x_{-1},x_0, x_{1},\dots$ is any bi-infinite sequence of points all of which lie in the same component of the space, then no matter how small the allowed (persistent) error, there is an approximate solution which runs through these points in sequence. Thus all knowledge of the asymptotic behavior is lost when (arbitrarily small) persistent errors are allowed.
\end{remark}

\begin{definition}
Let $\{M_1, \dots, M_n\}$ be any finite ordered collection of invariant sets with the following property: the sets are disjoint and on collapsing them to distinct points there is obtained a gradient-like quotient flow whose rest point set consists of the collapse points, and such that the ordering of the sets corresponds to that of the points induced by the Lyapounov function.
 Each of the sets $M_i$ is called a Morse set and the ordered collection is called a Morse decomposition.
\end{definition}


\subsubsection{Equivalence}
\paragraph{Bisimulation}
Bisimulation of dynamical systems is shown to be a concept which unifies the system-theoretic concepts of state space
equivalence and state space reduction, and which allows to study equivalence of systems with non-minimal state space dimension. 
%same computation, different implementations
The notion of bisimulation is especially powerful for ’non-deterministic’ dynamical systems, and leads in this case to a notion of equivalence which is finer than equality of external behavior.

\citep{vanderschaft2004bisimulation}

\citep{vanderschaft2004equivalence}

\subsection{Finite state machines}
%Synchronous sequential logic
%propagation delay Sec.\ref{sec:} in particular Def.~\ref{def:}

A deterministic finite-state machine or deterministic finite-state acceptor is a quintuple $(\Sigma ,S,s_{0},\delta,F)$, where: $\Sigma$  is the input alphabet (a finite non-empty set of symbols); $S$ is a finite non-empty set of states; $s_{0}$ is an initial state, an element of $S$ and $\delta$  is the state-transition function: 
$\delta :S\times \Sigma \rightarrow S$ (in a nondeterministic finite automaton it would be $\delta :S\times \Sigma \rightarrow {\mathcal {P}}(S)$, i.e. $\delta$  would return a set of states); $F$ is the set of final states, a (possibly empty) subset of $S$.


\begin{remark}
For both deterministic and non-deterministic FSMs, it is conventional to allow  $\delta$  to be a partial function, i.e. $\delta (s,x)$ does not have to be defined for every combination of $s\in S$ and $x\in \Sigma$. If an FSM $M$ is in a state $s$, the next symbol is $x$ and $\delta (s,x)$ is not defined, then $M$ can announce an error (i.e. reject the input). This is useful in definitions of general state machines, but less useful when transforming the machine. Some algorithms in their default form may require total functions.
\end{remark}


\begin{remark}
A finite-state machine has the same computational power as a Turing machine that is restricted such that its head may only perform "read" operations, and always has to move from left to right. That is, each formal language accepted by a finite-state machine is accepted by such a kind of restricted Turing machine, and vice versa.
\end{remark}


\subsubsection{Moore machines}% FSM + output

A Moore machine can be defined as a 6-tuple  $(S,s_{0},\Sigma ,O,\delta ,G)$ where $(\Sigma ,S,s_{0},\delta ,F)$ are the same as for an FSM and 
an output function $G:S\rightarrow O$ mapping each state to the output alphabet.



A Moore machine can be regarded as a restricted type of finite-state transducer.






%\begin{figure}[H]
% \centering
% \includegraphics[width=0.99\textwidth]{figs/.pdf}
% \caption{}
% \label{fig:tanh_dydot}
%\end{figure}



 \newpage
 \addcontentsline{toc}{section}{References}
 \bibliographystyle{ieeetr}
 \bibliography{ref.bib}

\end{document}
