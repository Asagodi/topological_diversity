\documentclass{article}
\usepackage[utf8]{inputenc}
\usepackage{amssymb, amsmath, amsthm}
\usepackage{thmtools, mathtools, mathrsfs}
\usepackage{amsfonts}
\usepackage[sort&compress,numbers]{natbib}
\usepackage{subcaption}
\usepackage{graphicx}
\usepackage{caption}
\usepackage{float}
\usepackage{bm}
\usepackage{tikz}
\usetikzlibrary{positioning,matrix,arrows,decorations.pathmorphing}
\usepackage{tikz-cd} 
\definecolor {processblue}{cmyk}{0.8,0,0,0}

\newcommand{\mb}[1]{\mathbb{#1}}
\newcommand{\mc}[1]{\mathcal{#1}}
\DeclareMathOperator{\Inv}{Inv}
\DeclareMathOperator{\innt}{int}
\newcommand{\probP}{\text{I\kern-0.15em P}}

\newtheorem{theorem}{Theorem}
\newtheorem{prop}{Proposition}
\theoremstyle{definition}
\newtheorem{definition}{Definition}
\theoremstyle{remark}
\newtheorem{remark}{Remark}

 \usepackage{thmtools, thm-restate} \newtheorem{conjecture}[theorem]{Conjecture}

\title{}
\author{\'Abel S\'agodi}
\date{September 1, 2023}

\begin{document}
\maketitle

\section{Infinite horizon computation}	
\begin{definition}
Infinite horizon: agnostic to decoding time
\end{definition}

\section{Theorems on infinite horizon neural computation}
\begin{theorem}
Neural computation that is bounded has to rely on asymptotic dynamics or on omega sets.
\end{theorem}

\begin{proof}
Invariance to time decoding implies that trajectories (after the input is presented) that get decoded into a certain output need to retain that decoding

Convergence to the non-wandering set for a compact dynamical system

Each of these trajectories will be in the same $\omega$-limit set
\end{proof}


%\begin{figure}[H]
% \centering
% \includegraphics[width=0.99\textwidth]{figs/.pdf}
% \caption{}
% \label{fig:tanh_dydot}
%\end{figure}


\begin{theorem}
For neural computation with a linear decoder, the sets of trajectories grouped by input must be linearly separable.
\end{theorem}


% \newpage
% \addcontentsline{toc}{section}{References}
% \bibliographystyle{plain}
% \bibliography{CIT-for-Computation.bib, cit.bib, CITCOD.bib}

\end{document}